
% ----- Misc non-math symbols ----------------------
\newcommand\COtwo{\ce{CO2}}
\DeclareSIUnit\kgcotwoe{kg\,CO_2e}
\sisetup{
  range-phrase=--,%
  round-pad=false,%
  round-mode=places,%
  round-precision=3,%
  per-mode=symbol,%
  % tables
  table-alignment-mode=format,%
  table-number-alignment=right,%
}

\newenvironment{nobreaks}{\vbox\bgroup}{\egroup}

% --------------------------------------------------------------
% PRELIMINARIES
% --------------------------------------------------------------

% ----- General math symbols ----------------------
% https://tex.stackexchange.com/a/43009
\DeclarePairedDelimiter\abs{\lvert}{\rvert}%
\DeclarePairedDelimiter\norm{\lVert}{\rVert}%

% Sets
\newcommand\N{\mathbb{N}}
\newcommand\Z{\mathbb{Z}}
\newcommand\R{\mathbb{R}}
\newcommand{\dcup}{\,\dot\cup\,}
\newcommand\Pot{\mathcal{P}}

% Functions
\newcommand{\pfunc}{\not\to}  % \nrightarrow looks different
\DeclareMathOperator{\dom}{dom}
\DeclareMathOperator{\rng}{rng}
\newcommand\restr[2]{{% we make the whole thing an ordinary symbol
\left.\kern-\nulldelimiterspace % automatically resize the bar with \right
#1 % the function
\vphantom{\big|} % pretend it's a little taller at normal size
\right|_{#2} % this is the delimiter
}}

% Multisets
% \newcommand{\B}{\mathcal{B}}  % multiset/bag notation
% \newcommand{\emptymultiset}{[\;]}

% Sequences
\newcommand{\emptysequence}{\langle \; \rangle}

% Graphs
\DeclareMathOperator{\Vertices}{V}
\DeclareMathOperator{\Edges}{E}
\DeclareMathOperator{\Comp}{C}
\newcommand{\dist}[1]{\operatorname{dist}_{#1}}
% \newcommand{\Vertices}[1]{\mathrm{V}\left( #1 \right)}
% \newcommand{\Edges}[1]{\mathrm{E}\left( #1 \right)}
% \newcommand{\Comp}[1]{\mathrm{C}\left( #1 \right)}
% \newcommand{\dist}[1]{\mathrm{dist}\left( #1 \right)}

% Statistics
\DeclareMathOperator{\mean}{mean}
\DeclareMathOperator{\median}{median}
\DeclareMathOperator{\std}{std}
\DeclareMathOperator{\cv}{cv}
% \DeclareMathOperator{\CV}{cv}  % TODO remove uppercase

% ----- PADS/Wil Symbols ----------------------

% ----- OCEL 2.0 symbols ----------------------
\newcommand{\Univ}[1]{\mathbb{U}_{\mathit{#1}}}
% \EA,\OA,\evtype,\time,\objtype,\eatype,\oatype,\eaval,\oaval,\E2O,\O2O
\newcommand{\EA}{\mathit{EA}}
\newcommand{\OA}{\mathit{OA}}
\newcommand{\ea}{\mathit{ea}}
\newcommand{\oa}{\mathit{oa}}
\newcommand{\evtype}{\mathit{evtype}}
% \newcommand{\time}{\mathit{time}}
\newcommand{\objtype}{\mathit{objtype}}
\newcommand{\eatype}{\mathit{eatype}}
\newcommand{\oatype}{\mathit{oatype}}
\newcommand{\eaval}{\mathit{eaval}}
\newcommand{\oaval}{\mathit{oaval}}
\newcommand{\EtoO}{\mathit{E2O}}
\newcommand{\OtoO}{\mathit{O2O}}
\newcommand{\ET}{\mathit{ET}}
\newcommand{\OT}{\mathit{OT}}
\newcommand{\ETL}{\mathit{ET}(L)}
\newcommand{\OTL}{\mathit{OT}(L)}
\newcommand{\et}{\mathit{et}}
\newcommand{\ot}{\mathit{ot}}
\newcommand{\QEtoOL}{Q_{\EtoO}(L)}
\newcommand{\QOtoOL}{Q_{\OtoO}(L)}

% ----- Cases&Variants notation ----------------------
% \newcommand{\trace}{\mathit{trace}}
% \newcommand{\traceL}{\mathit{trace}_L}
\newcommand{\obj}{\mathit{obj}}
\newcommand{\objL}{\mathit{obj}_L}
% \newcommand{\con}{\mathit{con}}
% \newcommand{\conL}{\mathit{con}_L}
% \newcommand{\ext}{\mathit{ext}}

% ----- Running example OCEL ----------------------
\newcommand{\itm}[1]{\texttt{i#1}}
\newcommand{\allitms}{\bigl\{ \itm{1}, \itm{2}, \itm{3}, \itm{4} \bigr\}}
\newcommand{\ev}[2]{\texttt{{\small #1\_}#2}}

% --------------------------------------------------------------
% MATH UTILS (not defined in preliminaries)
% --------------------------------------------------------------

\newcommand{\qqed}{\hfill$\square$}

% ----- \abs and \norm ----------------------
% Swap the definition of \abs* and \norm*, so that \abs
% and \norm resizes the size of the brackets, and the 
% starred version does not.
\makeatletter
\let\oldabs\abs
\def\abs{\@ifstar{\oldabs}{\oldabs*}}
%
\let\oldnorm\norm
\def\norm{\@ifstar{\oldnorm}{\oldnorm*}}
\makeatother

% ----- Nina's math utils ----------------------
%%%% Data types %%%%%
\newcommand*{\set}[1]{\ensuremath{\left\{ #1 \right\}}}
\newcommand*{\mset}[1]{\ensuremath{\left[ #1 \right]}}
\newcommand*{\seq}[1]{\ensuremath{\left< #1 \right>}}
% \newcommand*{\qty}[1]{\ensuremath{\llbracket #1 \rrbracket}}

% Symbols
% \def\init{\ensuremath{\blacktriangleright}}
% \def\nope{\perp}

%%%%% Common Expressions %%%%%
\def\cp{\ensuremath{\caps{CP}}}
\def\ocpn{\caps{OCPN}}

%% Event Logs %%
% \def\ocel{\ensuremath{L_{oc}}}
\def\ocel{\ensuremath{L}}
% \def\qel{\caps{QEL}}

% Log sets %
% \def\ot{\ensuremath{\caps{OT}}}
% \def\eo{\ensuremath{\caps{E2O}}}
% \def\oo{\ensuremath{\caps{O2O}}}
% \def\oatt{\ensuremath{\caps{OA}}}
% \def\eatt{\ensuremath{\caps{EA}}}
% \def\ei{\ensuremath{\caps{EI}}}
% \def\oi{\ensuremath{\caps{OI}}}
% \def\oa{\ensuremath{\caps{OA}}}
% \def\ea{\ensuremath{\caps{EA}}}
% \def\qr{\ensuremath{\caps{QR}}}
% \def\qc{\ensuremath{\caps{QC}}}
% \def\qp{\ensuremath{\caps{QP}}}
% \def\qe{\ensuremath{\caps{QE}}}
% \def\qa{\ensuremath{\caps{QA}}}
% \def\qn{\ensuremath{\caps{QN}}}

%% Sets %%
\def\ints{\Z}
\def\reals{\R}

% Variable Sets %
\newcommand*{\bag}[1]{\ensuremath{\mathcal{B}(#1)}}
% \newcommand*{\zq}[1]{\ensuremath{\mathcal{I}(#1)}}
% \newcommand*{\zqa}[1]{\ensuremath{\mathcal{I}^{q}(#1)}}
% \newcommand*{\zqp}[1]{\ensuremath{\mathcal{I^{+}}(#1)}}
% \newcommand*{\zqm}[1]{\ensuremath{\mathcal{I^{-}}(#1)}}
\newcommand*{\power}[1]{\ensuremath{\mathcal{P}(#1)}}
\newcommand*{\unvs}[1]{\ensuremath{\mathcal{U}_{#1}}}
% \newcommand{\colqty}[1]{\ensuremath{\mathcal{C}(#1)}}
% \newcommand{\colqtya}[1]{\ensuremath{\mathcal{C}^{q}(#1)}}
% \newcommand{\aggcp}[1]{\ensuremath{\Sigma^{cp}_{#1}}}
% \newcommand{\aggit}[1]{\ensuremath{\Sigma^{it}_{#1}}}

% --------------------------------------------------------------
% OWN CONTRIBUTIONS (starting in Method chapter)
% --------------------------------------------------------------

% ------ Emission rules --------------------
\newcommand{\EAnum}{\EA_{\text{num}}}
\newcommand{\OAnum}{\OA_{\text{num}}}
\newcommand{\FL}{\mathcal{F}(L)}
\newcommand{\REL}{\mathcal{R}^E(L)}
\newcommand{\REtoOL}{\mathcal{R}^{\EtoO}(L)}
\newcommand{\OTLunique}[1]{\OT^*_L(#1)}
\newcommand{\objLunique}[1]{\objL^*(#1)}

% ------ Object type classes notation --------------------
% \newcommand{\OTypeHU}{\mathit{OT}_{\mathrm{HU}}}
% \newcommand{\OTypeResource}{\mathit{OT}_{\mathrm{resource}}}
% \newcommand{\OTypeTHU}{\mathit{OT}_{\mathrm{THU}}}

\newcommand{\ObjHU}{O_{\mathrm{HU}}}
\newcommand{\ObjResource}{O_{\mathrm{resource}}}
% \newcommand{\ObjTHU}{O_{\mathrm{THU}}}
% \newcommand{\ObjTarget}{\Omega}

% ------ Object Interaction Graph --------------------
\newcommand{\OG}{\mathit{OG}}
\newcommand{\OGHU}{\OG_{\text{HU}}}
\newcommand{\OGOmegaL}{\OG^\Omega(L)}
\newcommand{\OGHUOmegaL}{\OGHU^\Omega(L)}
\newcommand{\OGxOmegaL}{\OG^{\Omega,\times}(L)}
\newcommand{\OGHUxOmegaL}{\OGHU^{\Omega,\times}(L)}
% \newcommand{\TGL}{\mathit{TG}(L)}

% ------ Emissions (lower-case / map-based) --------------------
% \newcommand{\em}{\mathit{em}}
\newcommand{\emE}{\mathit{em}^E}
\newcommand{\emO}{\mathit{em}^O}
\newcommand{\emEtoO}{\mathit{em}^{\EtoO}}
\newcommand{\emTot}{\mathit{em}^E_{\text{tot}}}
\newcommand{\emEtoOmega}{\mathit{em}^{\mathit{E2\Omega}}}
\newcommand{\emRem}{\mathit{em}^E_{\text{rem}}}
\newcommand{\emAlloc}[1]{\mathit{em}^{#1}}

% ------ Emission aggregation --------------------
\newcommand{\emL}{\mathit{em}^L}
\newcommand{\emAct}{\mathit{em}^{\text{act}}}

% ------ Allocation rules --------------------
\newcommand{\Omegadef}{\emptyset\neq\Omega\subseteq O}
\newcommand{\alphaAT}{\alpha_\text{AT}}
\newcommand{\alphaPT}{\alpha_\text{PT}}
\newcommand{\alphaCT}[1]{\alpha^{#1}_\text{CT}}
\newcommand{\alphaCTall}{\alpha_\text{CT}}
\newcommand{\alphaCTHU}{\alpha_\text{CT,HU}}
\newcommand{\allocruledef}{E \to \power{\Omega} {\setminus}\!\set{\emptyset}}

% ------ Allocation properties --------------------
\newcommand{\funiform}{f_{\text{uniform}}}
\newcommand{\funique}{f_{\text{unique}}}
\newcommand{\funiformalpha}{\funiform(\alpha)}
\newcommand{\funiquealpha}{\funique(\alpha)}

% ------ Context-specific --------------------
\newcommand{\activity}[1]{\textit{#1}}
\newcommand{\otype}[1]{\textit{#1}}
\newcommand{\allocrule}[1]{\textsc{#1}}

\newcommand\citeneeded{\textbf{[?]}}



\DeclareSIUnit{\tkm}{tkm}
\DeclareSIUnit{\pkm}{pkm}
\DeclareSIUnit{\passenger}{passenger}

\DeclareSIUnit{\Ckm}{\qty{100}{\km}}



