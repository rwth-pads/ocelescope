
In this thesis, we explored the potential of OCPM in CA, motivated by the growing need for organizations to report their GHG emissions.
As a novel approach, OCPM offers a more comprehensive analysis of processes involving multiple types of objects, addressing key limitations of classical PM.

Our contributions are both theoretical and practical.
On a theoretical level, emissions are integrated into an OCEL in two steps:
First, we propose a framework employing emission rules capable of estimating event and E2O emissions based on hand-selected emission factors.
Second, allocation to events and objects is applied in order to create various emission perspectives and gain sustainability-related insights.
Allocation to objects hereby determines carbon footprints along an object's lifecycle, aligning with LCA principles.
To this end, three different allocation rules are proposed.

As a practical contribution, we developed the OCEAn web application,
which serves as a frontend tool enabling users to apply these methods in practice.
OCEAn allows for the annotation of event logs with emission data, facilitates the execution of different allocation rules, and the export of enriched OCEL files for further use.

% Using publicly available OCELs, the performance of different allocation rules is compared .

After evaluating the emission rules w.r.t. runtime and different characteristics of the resulting emission distribution,
a clear recommendation is deduced:
Only the graph-based approach is capable of correctly allocating emissions to objects,
% We find that a graph-based approach allocates most precisely,
at the cost of high runtime when interactions of process resources are considered.
With the distinction of resources and HUs,
runtime is lowered by 93.1\% (11s, median).
In addition, more logical object relationships are captured, and allocation is more precise.
Pruning of the object interaction graph by removing edges between objects of the same type is suggested, reducing runtime further, while results are fully preserved on the test datasets.

For future directions, emission estimation can be extended by additional techniques such as the discovery of event intervals for the use in time-based emission rules.
Apart from that, sustainability and domain knowledge is needed for defining emission rules.
Future research should address this limitation, for example by automatic emission factor recommendation.

Evaluation shows the importance of the distinction of resources and HUs.
This partition, currently defined manually, exhibits clear patterns in terms of the lifecycle lengths and degrees in the object interaction graph.
However, we have identified outliers.
Therefore, automating this setting is desirable for deployment in industry,
but needs to be thoroughly addressed.

The object allocation framework can be equipped with further rules using directed graphs, allowing for footprint annotation of intermediate products in manufacturing.
Other enhancements could include the sequential execution of multiple allocation rules, making the framework more robust to different OCEL characteristics.
Furthermore, the option of using object attributes for weighted emission distribution has been discussed.

In summary, this work lays the groundwork for applying OCPM to emission assessments and carbon accounting.
It provides both theoretical tools and a practical implementation that allows enriching event data with emission values.
Next to direct utilization for CA and LCA, providing emission-enriched datasets supports further research in the application of PM for sustainability.
