This chapter introduces all concepts and notation that are used in the remainder of this thesis.
In \autoref{sec:prelim-math}, basic mathematical foundations are introduced,
followed by object-centric event logs (OCELs) in \autoref{sec:prelim-ocel}.

% ---------------------------------------------------------------------
% ---------------------------------------------------------------------
\section{Mathematical Foundations}
\label{sec:prelim-math}
% ---------------------------------------------------------------------
% ---------------------------------------------------------------------

% TODO check for usage!
\begin{outline}
  \1 $\N = \{1, 2, 3, \dots\}$ denotes the set of \textit{natural numbers}, $\N_0=\N\cup\{0\}$ the set of \textit{natural numbers including zero}, $\Z$ the set of \textit{integers} and $\R$ the set of \textit{real numbers}. \textit{Infinity} is denoted by $\infty$ and is not contained in any of the aforementioned sets.
  %
  \1 For a set $X$, $\Pot(X)$ denotes the \textit{powerset} of $X$, i.e., the set of subsets of $X$.
  A \textit{partition} of $X$ is a set of subsets $P\subseteq\Pot(X)$ s.t. the subsets in $P$ are pairwise disjoint ($X' {\cap} X'' = \emptyset\;\forall X',X''\in P$) and their union is $X$, i.e., ($\bigcup_{X'\in P} X' = X$).
  %
  \1 A \textit{function} $f\colon X \to Y$ assigns a value $f(x)\in Y$ to each $x\in X$.
  A \textit{partial function} $f\colon X \pfunc Y$ does not necessarily map all $x\in X$ to a value in $Y$.
  For those $x\in X$ where $f(x)$ is undefined, we write $f(x)=\bot$.
  
  For a function or partial function $f\colon X\to Y$,
  the \textit{range} of $f$ is given by $\rng(f)=\{f(x)\mid x\in X\}\setminus\{\bot\}$.
  The \textit{domain} of $f$ is denoted by $\dom(f)=\{x\in X \mid f(y)\neq\bot\}$.
  %
  The \textit{inverse image} of a set $Y'\subseteq Y$ under $f$ is defined as
  $f^{-1}(Y) = \{ x\in X \mid f(x) \in Y \}$,
  making $f^{-1}$ a function $\Pot(Y)\to\Pot(X)$.
  %
  \1 We denote the \textit{minimum}
  %and {maximum}
  of a finite set $Y\subseteq\R\cup\{\infty\}$
  by $\min Y$.
  % and $\max X$, respectively.
  We assume $\min\emptyset=\infty$.
  % and $\max\emptyset=-\infty$.

  For sets $X$ and $Y\subseteq\R\cup\{\infty\}$, a function $f\colon X\to Y$ and a finite set $X'\subseteq X$, we write
  \begin{align*}
    \argmin_{x\in X'} f(x) &\coloneqq
    f^{-1}\Bigl( \Bigl\{ \min\{f(x) \mid x\in X'\} \Bigr\} \Bigr).
  \end{align*}
  %
  \1 A \textit{sequence} of length $n\in\N_0$ over a domain set $X$ is an ordered list containing elements $x_1,\dots,x_n\in X$ and is denoted by $\sigma=\seq{x_1,\dots,x_n}$. The length of a sequence is $\abs{\sigma}\coloneqq n$. The set of sequences over $X$ is denoted by $X^*$. The set of sequences of a specific length $n\in\N_0$ over $X$ is denoted by $X^n$. Sequences can be empty, with $X^0 = \{\emptysequence\}$.
  %
  \1 An \textit{undirected graph} is a tuple $G=(V,E)$
  where $\Vertices(G)\coloneqq V$ is the set of vertices
  and $\Edges(G)\coloneqq E\subseteq\bigl\{ \{u,v\}\subseteq V \bigm| u \neq v \bigr\}$ is the set of edges.
    %
    \2 Two vertices $u,v\in V$ are \textit{connected} in $G$ if $\{u,v\}\in E$.
    %
    \2 A \textit{path} of length $n\in\N_0$ from $v_0$ to $v_n$ in $G$ is a sequence of vertices $\seq{v_0, \dots, v_n}\in V^{n+1}$ where $\{v_i,v_{i+1}\}\in E$ for all $0 \leq i \leq n$.
    %
    \2 A vertex $v\in V$ is \textit{reachable} in $G$ from $u\in V$ if there exists a path from $u$ to $v$ in $G$.
    %
    \2 A \textit{connected component} of $G$ is a maximal set $V'\subseteq V$ s.t. all vertices in $V'$ are mutually reachable. The set of connected components of $G$ is a partition of $V$ and is denoted by $\Comp(G)$.
    %
    \2 The \textit{distance} of $u$ and $v$ in $G$ is the minimal $n\in\N_0$ s.t. there exists a path of length $n$ from $u$ to $v$ in $G$ and is denoted by $\dist{G}(u,v)\coloneqq n$. If $v$ is not reachable from $u$ in $G$, then $\dist{G}(u,v)\coloneqq \infty$. This makes $\dist{G}$ a function $V {\times} V \to \N_0 \cup\{\infty\}$.
    Note that $\dist{G}(u, v)=0$ if and only if $u=v$.
    %
    \2 The \textit{neighborhood} of $v\in V$ is the set of vertices $v$ is connected to. The \textit{degree} of $v$ is the size of the neighborhood of $v$ and is denoted by
    $\deg_G(v)\coloneqq\abs{\bigl\{ u \in V \bigm| \{u,v\}\in \Edges(G) \bigr\}}$
    %
    \2 A \textit{clique} in $G$ is a set of vertices that are mutually connected.
  %

\0 For data visualization and evaluation purposes, some descriptive statistics are used. These measures are defined for a given sequence of numbers
$x = \seq{x_1,\dots,x_n}\in\R^n$, $n\in\N$.
  \1 The \textit{mean} of $x$ is defined as
  $\mean(x)\coloneqq\frac{1}{n} \sum_{i=1}^n x_i$.
  % \2 The minimum and maximum of $x$ are defined as
  % $\min(x)\coloneqq \min\{x_1,\dots,x_n\}$ and
  % $\max(x)\coloneqq \max\{x_1,\dots,x_n\}$.
  %
  \1 For an ordering $x^{(1)} \leq\dots\leq x^{(n)}$ of $x$, the \textit{median} of $x$ is denoted by
  \begin{align*}
    \median(x) &\coloneqq \begin{cases}
      x^{((n+1)/2)} & \text{if $n$ is odd}, \\
      \frac{1}{2} \Bigl(
        x^{(n/2)} + x^{(n/2+1)}
      \Bigr) & \text{if $n$ is even}.
    \end{cases}
  \end{align*}
  %
  \1 For $n>1$, the \textit{standard deviation} of $x$ is defined as
  \begin{align*}
    \std(x) &\coloneqq \sqrt{
      \tfrac{1}{n-1} \sum_{i=1}^n \bigl(x_i - \mean(x)\bigr)^2
    }.
  \end{align*}
  The \textit{variation coefficient} is a relative version of the standard deviation and is denoted by
  $\cv(x)\coloneqq\frac{\std(x)}{\mean(x)}$.

\0 In the remainder of the thesis, the compact notation ``$m\,[a - b]$'' is used to summarize a number series $x\in\R^*$ with $\median(x)=m,\min(x)=a$ and $\max(x)=b$.
\end{outline}

% ---------------------------------------------------------------------
% ---------------------------------------------------------------------
\section{Object-Centric Event Logs}
\label{sec:prelim-ocel}
% ---------------------------------------------------------------------
% ---------------------------------------------------------------------

As mentioned in \autoref{chap:rw}, there have been two versions of the OCEL standard as of now.
This project's concept is built around the most recent one of them, the OCEL~2.0 standard~\cite{OCEL2}.
In the following, the formal definition of an OCEL according to the standard is given.

First, universe notation is used to denote a set of unique identifiers for events, objects and other entities.
The following universes are defined as pairwise disjoint sets:
\begin{itemize}
  \item $\Univ{ev}$ is the universe of events,
  \item $\Univ{etype}$ is the universe of event types,
  \item $\Univ{obj}$ is the universe of objects,
  \item $\Univ{otype}$ is the universe of object types,
  \item $\Univ{attr}$ is the universe of attribute names,
  \item $\Univ{qual}$ is the universe of qualifiers.
\end{itemize}
Additionally,
\begin{itemize}
  \item $\Univ{val}$ is the universe of attribute values,
  \item $\Univ{time}$ is the universe of timestamps, including $0,\infty\in\Univ{time}$, totally ordered with $0\leq t \leq \infty$ for all $t\in\Univ{time}$.
\end{itemize}

\begin{definition}
  An Object-Centric Event Log (OCEL) is a tuple
  \begin{align*}
    L &= (E,O,\EA,\OA,\evtype,\mathit{time},\objtype,\eatype,\oatype,\eaval,\oaval,\EtoO,\OtoO)
  \end{align*}
  where
  \begin{itemize}
    \item $E\subseteq\Univ{ev}$ is the set of events,
    \item $O\subseteq\Univ{obj}$ is the set of objects,
    \item $\evtype\colon E \to \Univ{etype}$ assigns event types to events,
    \item $\mathit{time}\colon E \to \Univ{time}$ assigns timestamps to events,
    \item $\objtype\colon O \to \Univ{otype}$ assigns object types to objects,
    \item $\EA\subseteq\Univ{attr}$ is the set of event attributes,
    \item $\OA\subseteq\Univ{attr}$ is the set of object attributes,
    \item $\eatype\colon \EA \to \Univ{etype}$ assigns event types to event attributes,
    \item $\oatype\colon \OA \to \Univ{otype}$ assigns object types to object attributes,
    \item $\eaval\colon(E {\times} \EA) \pfunc \Univ{val}$ assigns values to event attributes,
    \item $\oaval\colon(O {\times} \OA {\times} \Univ{time}) \pfunc \Univ{val}$ assigns values to object attributes,
    \item $\EtoO \subseteq E {\times} \Univ{qual} {\times} O$ are the qualified event-to-object relations,
    \item $\OtoO \subseteq O {\times} \Univ{qual} {\times} O$ are the qualified object-to-object relations.
  \end{itemize}
  \label{def:ocel}
\end{definition}

In the following, we define auxiliary notation for an OCEL $L$. Parts of these are originally proposed by the OCEL~2.0 standard:
\begin{itemize}
  \item $\ETL = \{\evtype(e) \mid e \in E\}$ is the set of event types (activities),
  \item $\OTL = \{\objtype(o) \mid o \in O\}$ is the set of object types,
  %
  % \item $\QEtoOL = \{q \in\Univ{qual} \mid \exists e,o: (e,q,o)\in\EtoO \}$ is the set of event-to-object qualifiers,
  %
  % \item $\QOtoOL = \{q \in\Univ{qual} \mid \exists o_1,o_2: (o_1,q,o_2)\in\OtoO \}$ is the set of object-to-object qualifiers,
  %
  % $\objL(e)$
  \item For an event $e\in E$, the set of objects participating in $e$ is denoted by
  \begin{align*}
    \objL(e) &\coloneqq \{o \in O \mid \exists q\in\Univ{qual}\colon(e, q, o)\in\EtoO \}.
  \end{align*}
  %
  % eaval aux
  % \item For an event $e\in E$ and an event attribute $\ea\in\Univ{attr}$ with $\evtype(e)=\eatype(\ea)$:
  % \begin{align*}
  %   \eaval _{\ea}(e) &\coloneqq\left\{\begin{array}{ll}
  %     \eaval(e, \ea)
  %     & \text{if } (e, \ea) \in\dom(\eaval), \\
  %     \bot & \text{else}.
  %   \end{array}\right. \\
  %   &\hspace*{30em}
  % \end{align*}
  % \vspace*{-3.5em}
  %
  % oaval aux
  % \item For an object $o\in O$, an object attribute $\oa\in\Univ{attr}$ with $\objtype(o)=\oatype(\oa)$, and a timestamp $t\in\Univ{time}$:
  % \begin{align*}
  %   \oaval_{\oa}^t(o) &\coloneqq\left\{\begin{array}{ll}
  %     \oaval(o, \oa, t')
  %     & \begin{array}{l}
  %       \exists t'\in\Univ{time} \colon t'\leq t \land (o, \oa, t') \in\dom(\oaval), \\
  %       \land \neg\exists t''\in\Univ{time} \colon \\
  %       \;\bigl(
  %         t' < t'' \leq t
  %         \land (o, \oa, t') \in\dom(\oaval)
  %       \bigr)
  %     \end{array}\\
  %     \bot & \text{else},
  %     % \hspace*{19em}
  %   \end{array}\right. \\
  %   \oaval_{\oa}(o) &\coloneqq \oaval_{\oa}^\infty(o). \\
  %   &\hspace*{30em}
  % \end{align*}
  % \vspace*{-3.5em}
  %
\end{itemize}

% ---------------------------------------------------------------------
\subsection{Example Log}
\label{ssec:prelim-ocel-rex}
% ---------------------------------------------------------------------

\begin{table}[t]
  \caption{An OCEL from a logistics process, represented by three tables: On the left, events and objects are given with identifiers, activity resp. object type, and further attributes. On the right, the E2O relations are listed, i.e., participations of objects in events.}
  \label{tab:rex-ocel}
  \centering
  \small
  \begin{minipage}[t]{.65\textwidth}
    \strut\vspace*{-\baselineskip}\newline
    % Events
    \begin{tabularx}{\textwidth}{p{2cm}p{2cm}S[table-format=3.1,table-number-alignment=center]X}
      \toprule
      \textbf{Event} & \textbf{Activity} & \textbf{Distance} & \textbf{Timestamp} \\
      \midrule
      \ev{receive}{o1} & receive order & {\textemdash} & 05-07 21:00 \\
      \ev{get}{i1} & get item & {\textemdash} & 05-10 18:00 \\
      \ev{receive}{o2} & receive order & {\textemdash} & 05-14 10:30 \\
      \ev{load}{p1} & load & {\textemdash} & 05-14 12:00 \\
      \ev{deliver}{1} & deliver & 52.00 & 05-14 13:00 \\
      \ev{get}{i3} & get item & {\textemdash} & 05-15 13:00 \\
      \ev{get}{i2} & get item & {\textemdash} & 05-20 13:00 \\
      \ev{get}{i4} & get item & {\textemdash} & 05-20 13:10 \\
      \ev{load}{p2} & load & {\textemdash} & 05-20 14:00 \\
      \ev{load}{p3} & load & {\textemdash} & 05-20 14:05 \\
      \ev{deliver}{p2p3} & deliver & 20.00 & 05-20 15:00 \\
      \ev{deliver}{p3} & deliver & 40.00 & 05-20 15:42 \\
      \ev{inspect}{v1} & inspect & 118.00 & 05-22 15:00 \\
      \ev{inspect}{v2} & inspect & 115.00 & 05-26 16:30 \\
      \bottomrule
    \end{tabularx}
    
    \vspace*{.25em}
    % \vfill

    % Objects
    \begin{tabularx}{\textwidth}{p{2cm}p{2cm}S[table-format=2.1,table-number-alignment=center]X}
      % \toprule
      \textbf{Object} & \textbf{Obj. type} & \textbf{Consumption} & \\
      \midrule
      \texttt{o1} & Order & {\textemdash} &  \\
      \texttt{o2} & Order & {\textemdash} &  \\
      \texttt{i1} & Item & {\textemdash} &  \\
      \texttt{i2} & Item & {\textemdash} &  \\
      \texttt{i3} & Item & {\textemdash} &  \\
      \texttt{i4} & Item & {\textemdash} &  \\
      \texttt{p1} & Pallet & {\textemdash} &  \\
      \texttt{p2} & Pallet & {\textemdash} &  \\
      \texttt{p3} & Pallet & {\textemdash} &  \\
      \texttt{v1} & Vehicle & 10 &  \\
      \texttt{v2} & Vehicle & 12.5 &  \\
      \bottomrule
    \end{tabularx}
  \end{minipage}
  \hspace*{1em}
  \begin{minipage}[t]{.3\textwidth}%\rule{2cm}{1cm}
    \strut\vspace*{-\baselineskip}\newline
    % E2O
    \begin{tabular}{ll}
      \toprule
      \textbf{Event} & \textbf{Object} \\
      \midrule
      \ev{receive}{o1} & \texttt{o1} \\
      \ev{get}{i1} & \texttt{o1} \\
      \ev{get}{i1} & \texttt{i1} \\
      \ev{receive}{o2} & \texttt{o2} \\
      \ev{load}{p1} & \texttt{i1} \\
      \ev{load}{p1} & \texttt{p1} \\
      \ev{load}{p1} & \texttt{v1} \\
      \ev{deliver}{1} & \texttt{p1} \\
      \ev{deliver}{1} & \texttt{v1} \\
      \ev{get}{i3} & \texttt{o2} \\
      \ev{get}{i3} & \texttt{i3} \\
      \ev{get}{i2} & \texttt{o1} \\
      \ev{get}{i2} & \texttt{i2} \\
      \ev{get}{i4} & \texttt{o2} \\
      \ev{get}{i4} & \texttt{i4} \\
      \ev{load}{p2} & \texttt{i2} \\
      \ev{load}{p2} & \texttt{p2} \\
      \ev{load}{p2} & \texttt{v2} \\
      \ev{load}{p3} & \texttt{i3} \\
      \ev{load}{p3} & \texttt{i4} \\
      \ev{load}{p3} & \texttt{p3} \\
      \ev{load}{p3} & \texttt{v2} \\
      \ev{deliver}{p2p3} & \texttt{p2} \\
      \ev{deliver}{p2p3} & \texttt{p3} \\
      \ev{deliver}{p2p3} & \texttt{v2} \\
      \ev{deliver}{p3} & \texttt{p3} \\
      \ev{deliver}{p3} & \texttt{v2} \\
      \ev{inspect}{v1} & \texttt{v1} \\
      \ev{inspect}{v2} & \texttt{v2} \\
      \bottomrule
    \end{tabular}
  %
  \end{minipage}
\end{table}

\autoref{tab:rex-ocel} shows an exemplary OCEL.
The dataset corresponds to the same logistics process introduced in \autoref{sec:intro-motiv}: The company receives \otype{orders} consisting of multiple \otype{items}. These items get loaded onto \otype{pallets} and are then delivered using \otype{vehicles}.

The data are shown by means of three tables: events, objects, and E2O relations.
As opposed to other formats like OCEL 1.0~\cite{OCEL1}, the second version of the standard uses this relational data model to efficiently store object-centric process data in an \texttt{sqlite} database file, avoiding sparse tables.

The events (top left table) are represented by an identifier and are assigned an activity, a timestamp, and more attributes, in this case a distance (for the activities \activity{deliver} and \activity{inspect}).
Objects (bottom left) have an identifier and a type, with \otype{vehicles} having the attribute \textit{consumption}.
E2O relations (right) are pairs of one event and one object.
For simplicity, the example does not contain qualified relations.

\autoref{chap:method} introduces the method employed for emission analysis of OCELs.
For this, the logistics event log is used as a minimal example to demonstrate different parts of the method.

% ---------------------------------------------------------------------
\subsection{Object Interaction Graphs}
\label{ssec:prelim-ocel-og}
% ---------------------------------------------------------------------

In OCPM, the object interaction graph captures relations between pairs of objects extracted from the log.
Two objects have are connected in this graph if they share some common event~\cite{Fahland22process,Adams22defining}.
With the advent of OCEL~2.0~\cite{OCEL2}, object interaction graphs have been enriched with the newly added O2O relations, or vice versa~\cite{Liss24totem}.
This concept is used in this work as well,
allowing to explictly add more edges to the graph if two objects do not interact in an event.

Object interaction graphs are used in this work for allocating emissions to a subset of the objects, called target objects.
To this end, shortest paths from other objects to target objects are computed.
As edges between pairs of target objects are not relevant for this use case, they are removed in order to reduce graph size.

\begin{nobreaks}
  \begin{definition}[Object Interaction Graph]
    For an OCEL~$L$ and a set of target objects~$\Omega\subseteq O$ with~$\Omega\neq\emptyset$, the \textit{object interaction graph} $\OGOmegaL$ is an undirected graph defined by
    \begin{alignat*}{6}
      \Vertices\bigl(\OGOmegaL\bigr) &\,&=&&\,& O, \\
      \Edges\bigl(\OGOmegaL\bigr) &&=&&&
        \Bigl(\Bigl\{
          \{ o_1, o_2 \} \subseteq O
        \Bigm|
          o_1 {\neq} o_2 \land
          \exists e\in E \colon o_1,o_2 \in\objL(e)
        \Bigr\} \\
        &&\cup&&&
        \hphantom{\Big(} \Bigl\{
          \{o_1,o_2\} \subseteq O
        \Bigm|
          o_1 {\neq} o_2 \land
          (o_1,q,o_2)\in\OtoO
        \Bigr\} \Bigr) \\
        % \;\setminus\; \Pot(\Omega)
        &&\setminus&&& \Pot(\Omega)
    \end{alignat*}
    \label{def:og}
  \end{definition}
\end{nobreaks}

\begin{figure}[t]
	\centering
  \small
  \begin{tikzpicture}[
    event/.style={rectangle,draw,inner sep=0,minimum size=.6cm},
    object/.style={circle,draw,inner sep=.25em,font=\ttfamily},
    target object/.style={object,double},
    order/.style={object,fill=order!25},
    item/.style={target object,fill=item!25},
    pallet/.style={object,fill=pallet!25},
    vehicle/.style={object,fill=vehicle!25},
  ]
    \node[item] (i1) {i1};
    \node[item,right=1cm of i1] (i2) {i2};
    \node[item,right=2cm of i2] (i3) {i3};
    \node[item,right=1cm of i3] (i4) {i4};
    \coordinate (n_i1_i2) at ($(i1.north)!.5!(i2.north)$);
    \coordinate (n_i3_i4) at ($(i3.north)!.5!(i4.north)$);
    \node[order,above=.75cm of n_i1_i2] (o1) {o1};
    \node[order,above=.75cm of n_i3_i4] (o2) {o2};
    \node[pallet,below=.75cm of i1] (p1) {p1};
    \node[pallet,below=.75cm of i2] (p2) {p2};
    % \node[pallet] (p3) at ($(i3)!.5!(i4) + (0,1cm)$) {p\_3};
    \coordinate (s_i3_i4) at ($(i3.south)!.5!(i4.south)$);
    \node[pallet,below=.75cm of s_i3_i4] (p3) {p3};
    \node[vehicle,below=.75cm of p1] (v1) {v1};
    \coordinate (s_p2_p3) at ($(p2.south)!.5!(p3.south)$);
    \node[vehicle,below=.75cm of s_p2_p3] (v2) {v2};
    %
    \draw
      (o1) edge (i1) (o1) edge (i2)
      (o2) edge (i3) (o2) edge (i4)
      (i1) edge (p1)
      (p1) edge (v1)
      (i2) edge (p2)
      (i3) edge (p3) (i4) edge (p3)
      (p2) edge (v2) (p3) edge (v2)
      (i2) edge (v2) (i3) edge (v2)
      (p2) edge (p3);
    \draw[bend left=40] (i4) edge (v2);
    \draw[bend right=60] (i1) edge (v1);
    \draw[rwthred,dashed] (i3) -- (i4)
      node[draw,solid,strike out,color=rwthred,pos=.45] {}
      node[draw,solid,strike out,color=rwthred,pos=.55] {};
      % node[draw,strike out,color=rwthred,pos=.55] {};
  \end{tikzpicture}
	\caption{Object interaction graph of the logistics example log. For visualization, objects are colored depending on their type, target objects (here all \otype{items}) are highlighted with double borders. The \otype{items} \texttt{i3} and \texttt{i4} are \textbf{not} connected as they are both target objects.}
	\label{fig:og-example}
\end{figure}

\autoref{fig:og-example} shows the object interaction graph $\OGOmegaL$ extracted from the example log from \autoref{tab:rex-ocel}, with
$\Omega\coloneqq\{ \texttt{i1}, \texttt{i2}, \texttt{i3}, \texttt{i4} \}$ defined as the set of target objects.
As the OCEL does not contain O2O relations, the edges all correspond to object interactions.
There is no edge between the \otype{item} objects \texttt{i3} and \texttt{i4} as they are both target objects. When instead labeling the \otype{pallet} objects as targets, the items will be connected, and the edge betweeen \texttt{p2} and \texttt{p3} will be removed.
